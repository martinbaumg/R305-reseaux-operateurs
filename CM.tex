\documentclass[12pt, a4paper]{article}
\usepackage[francais]{babel}
\usepackage{caption}
\usepackage{graphicx}
\usepackage[T1]{fontenc}
\usepackage{listings}
\usepackage{geometry}
\usepackage{minted}
\usepackage{array,multirow,makecell}
\usepackage[colorlinks=true,linkcolor=black,anchorcolor=black,citecolor=black,filecolor=black,menucolor=black,runcolor=black,urlcolor=black]{hyperref}
\setcellgapes{1pt}
\makegapedcells
\usepackage{fancyhdr}
\pagestyle{fancy}
\lhead{}
\rhead{}
\chead{}
\rfoot{\thepage}
\lfoot{Martin Baumgaertner}
\cfoot{}
\renewcommand{\footrulewidth}{0.4pt}
\renewcommand{\headrulewidth}{0.4pt}
\renewcommand{\listingscaption}{Code}
\renewcommand{\listoflistingscaption}{Table des codes}
% \usepackage{mathpazo} --> Police à utiliser lors de rapports plus sérieux

\begin{document}
\begin{titlepage}
	\newcommand{\HRule}{\rule{\linewidth}{0.5mm}} 
	\center 
	\textsc{\LARGE iut de colmar}\\[1.5cm] 
	\textsc{\Large R305}\\[0.5cm] 
	\textsc{\large Année 2022-23}\\[0.5cm]
	\HRule\\[0.75cm]
	{\huge\bfseries Réseaux opérateurs}\\[0.4cm]
	\HRule\\[1.5cm]
	\textsc{\large martin baumgaertner}\\[0.5cm] 

	\vfill\vfill\vfill
	{\large\today} 
	\vfill
\end{titlepage}
\newpage
\tableofcontents
\newpage
\section{CM 1 - 2 septembre 2022}
\subsection{Introduction}
    \begin{itemize}
        \item La couche d’accès fourni un moyen de connecter des périphériques au réseau. 
        \item La couche de distribution gère le flux du traffic réseau à l’aide de 
        stratégie politiques de distribution.
        \item la couche coeur consitutye lé réseau fédérateur hait débit de 
        l’inter-réseau.\\
    \end{itemize}
Il y a deux catégories de matériel utilisé pour interconnecter :

IGP : Protocoles de routage utilisés à l'intérieur d'un système autonome

EGP : Protocole de riytage utilisé pour échanger des informations de routage
entre différents système autonomes.\\

BGP à deux variantes :

    \begin{itemize}
        \item External BGP (eBGP) : utilisé enrre des AS différents 
        (distance administrative : 20)
        \item Internal BGP (iBGP) : utilisé pour interconnecter des AS
        (distance administrative : 200)
    \end{itemize}
  
\subsection{Pourquoi utiliser BGP}
    Un IGP n'est pas suffisant car si un routeur doit échanger des informations
    sur toutes les routes du monde, donc à l'aide du BGP, il est conçu pour
    créer des sessions de routage entre les différents routeurs.\\

    Si par exemple un client possède plusieurs serveurs et qu'il veut mettre 
    tous ses services sur internet, qu'il a donc la même adresse IP, on va 
    donc faire du port forwarding\\

    Si maintenant, on veut assurer de la redondance, et qu'on veut un lien
    primaire et autre seconfaire avec un équilibrage de charge 50/50, on aura
    toujours pas besoin de BGP. \\
    
    Le cas d'usage principal de BGP, c'est d'interconnecter, pour avoir une meilleure
    redondance. 
    BGP permet de faire transiter les routes publiques vers les AS.
    \newpage

    \subsection{BGP - concetps de base}
    \begin{itemize}
        \item BGP est un protocole de routage à vecteur de chemin
        \item Fait les décisions selon les politiques de routage
        \item N'indique pas les détails internes des AS
        \item Ne représente qu'un arbre des AS
        
    \end{itemize}



\end{document}