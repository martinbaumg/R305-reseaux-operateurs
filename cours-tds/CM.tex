\documentclass[12pt, a4paper]{article}
\usepackage[francais]{babel}
\usepackage{caption}
\usepackage{graphicx}
\usepackage{bigfoot}
\usepackage[T1]{fontenc}
\usepackage{listings}
\usepackage{geometry}
\usepackage{minted}
\usepackage{array,multirow,makecell}
\usepackage[colorlinks=true,linkcolor=black,anchorcolor=black,citecolor=black,filecolor=black,menucolor=black,runcolor=black,urlcolor=black]{hyperref}
\setcellgapes{1pt}
\makegapedcells
\usepackage{fancyhdr}
\pagestyle{fancy}
\lhead{}
\rhead{}
\chead{}
\rfoot{\thepage}
\lfoot{Martin Baumgaertner}
\cfoot{}
\renewcommand{\footrulewidth}{0.4pt}
\renewcommand{\headrulewidth}{0.4pt}
\renewcommand{\listingscaption}{Code}
\renewcommand{\listoflistingscaption}{Table des codes}
% \usepackage{mathpazo} --> Police à utiliser lors de rapports plus sérieux

\begin{document}
\begin{titlepage}
	\newcommand{\HRule}{\rule{\linewidth}{0.5mm}} 
	\center 
	\textsc{\LARGE iut de colmar}\\[6.5cm] 
	\textsc{\Large R305}\\[0.5cm] 
	\textsc{\large Année 2022-23}\\[0.5cm]
	\HRule\\[0.75cm]
	{\huge\bfseries Réseaux opérateurs}\\[0.4cm]
	\HRule\\[1.5cm]
	\textsc{\large martin baumgaertner}\\[6.5cm] 

	\vfill\vfill\vfill
	{\large\today} 
	\vfill
\end{titlepage}
\newpage
\tableofcontents
\listoflistings
\newpage
\section{CM 1 - 2 septembre 2022}
\subsection{Introduction}
    \begin{itemize}
        \item La couche d’accès fourni un moyen de connecter des périphériques au réseau. 
        \item La couche de distribution gère le flux du traffic réseau à l’aide de 
        stratégie politiques de distribution.
        \item la couche coeur consitutye lé réseau fédérateur hait débit de 
        l’inter-réseau.\\
    \end{itemize}
Il y a deux catégories de matériel utilisé pour interconnecter :

IGP : Protocoles de routage utilisés à l'intérieur d'un système autonome

EGP : Protocole de riytage utilisé pour échanger des informations de routage
entre différents système autonomes.\\

BGP à deux variantes :

    \begin{itemize}
        \item External BGP (eBGP) : utilisé enrre des AS différents 
        (distance administrative : 20)
        \item Internal BGP (iBGP) : utilisé pour interconnecter des AS
        (distance administrative : 200)
    \end{itemize}
  
\subsection{Pourquoi utiliser BGP}
    Un IGP n'est pas suffisant car si un routeur doit échanger des informations
    sur toutes les routes du monde, donc à l'aide du BGP, il est conçu pour
    créer des sessions de routage entre les différents routeurs.\\

    Si par exemple un client possède plusieurs serveurs et qu'il veut mettre 
    tous ses services sur internet, qu'il a donc la même adresse IP, on va 
    donc faire du port forwarding\\

    Si maintenant, on veut assurer de la redondance, et qu'on veut un lien
    primaire et autre seconfaire avec un équilibrage de charge 50/50, on aura
    toujours pas besoin de BGP. \\
    
    Le cas d'usage principal de BGP, c'est d'interconnecter, pour avoir une meilleure
    redondance. 
    BGP permet de faire transiter les routes publiques vers les AS.
    \newpage

    \subsection{BGP - concetps de base}
    \begin{itemize}
        \item BGP est un protocole de routage à vecteur de chemin
        \item Fait les décisions selon les politiques de routage
        \item N'indique pas les détails internes des AS
        \item Ne représente qu'un arbre des AS
        
    \end{itemize}
\newpage  
\section{TD 1 - 7 septembre 2022}
\subsection{Exercice 1}
    \subsubsection{Question 1}
    L'IGP est utilisé pour l'intérieur d'un système tandis que EGP est utilisé
    pour échanger des informations entre différents systèmes autonomes. 
    La capacité de traitement est aussi différente, BGP a une plus grande capacité de 
    traitement. 
    
    \subsubsection{Question 2}
    \texttt{Path Vector} signifie que c'est un protocole de routage à vecteur de 
    chemin qui conserve les informations de chemin qui sont mises à jour dynamiquement. 

    \subsubsection{Question 3}
    Le numéro de port utilisé pour BGP est le port 179. Le mode de connexion utilisé
    pour ses sessions est TCP. 

    \subsubsection{Question 4}
    La détection de boucle se fait grâce aux numéros d'AS, un système autonome peut
    détecter un chemin qui passe deux fois par le même AS. 

    \subsubsection{Question 5}
    La version actuelle de BGP est \texttt{BGP4}.

    \subsubsection{Question 6}
    La stratégie de base appliquée par les routeurs BGP pour choisir le chemin
    vers la destination est la stratégie de distance administrative.

    \subsubsection{Question 7}
    Un peering est une paire de routeurs qui établissent une session BGP avec
    connexion TCP. 

    \subsubsection{Question 8}
    Faux. 

    \subsubsection{Question 9}
    Le rôle des attributs suivants : 
    \begin{itemize}
        \item \texttt{Local pref} : permet de définir par quel routeur on veut recevoir les données et par quel routeur on veut sortir les données
        \item \texttt{MED} : permet de définir la priorité du chemin de réception. 
    \end{itemize}

    \subsubsection{Question 10}
    Le deux variantes de BGP sont :
    \begin{itemize}
        \item \texttt{eBGP} : utilisé entre des AS différents
        \item \texttt{iBGP} : utilisé pour interconnecter des AS
    \end{itemize}

    \subsubsection{Question 11}
    Nous l'utiliserons si on veut contrôler la réparition du flux. Mais aussi
    pour l'interconnexion de systèmes autonomes. 

    \subsubsection{Question 12}
    Non pas forcément, le partage peut être partiel au total. 

    \subsubsection{Question 13}
    Il existe actuellement plus de 900 000 préfixes réseaux. 

\subsection{Exercice 2}
    \subsubsection{Question 1}
    R2 : \\ 
    C.192.16.1.0/24\\
    C.11.1.2.0/24\\

    R3 : \\
    C.12.1.0.0/24\\

    R5 : \\
    C.128.213.11.0/24\\
    C.212.1.1.0/24\\
    R.172.16.0.0/24\\
    R.10.0.0.0/24\\

    B3 : \\
    C.1.1.1.0/24\\
    C.128.213.11.0/24\\
    R.212.1.1.0/24\\
    R.10.0.0.0/24\\
    R.172.16.0.0/24\\

    \subsection{Exerice 3}
    \subsubsection{Question 1}
    Voici les configurations des différents routeurs. 
    \begin{listing}[H]
        \caption{Configuration du routeur 1 - OSPF}
        \label{lst:r1o}
        \begin{minted}{bash}
        router OSPF 1 
        network 172.16.12.0 0.0.0.3
        network 192.168.100.1 0.0.0.0
        interface lo 0
        ip add 192.168.100.1 255.255.255.255
        \end{minted}
    \end{listing}

    \begin{listing}[H]
        \caption{Configuration du routeur 1 - BGP}
        \label{lst:r1b}
        \begin{minted}{bash}
        router BGP 100
        neighbor 192.168.100.3 remote-as 100
        \end{minted}
    \end{listing}

    \begin{listing}[H]
        \caption{Configuration du routeur 2 - BGP}
        \label{lst:r2b}
        \begin{minted}{bash}
        router BGP 100
        neighbor 192.168.100.1 remote-as 100
        neighbor 192.168.100.1 update-source loopback 0
        neighbor 192.168.100.3 remote-as 100
        neighbor 192.168.100.3 update-source loopback 0
        \end{minted}
    \end{listing}

    \begin{listing}[H]
        \caption{Configuration du routeur 3 - BGP}
        \label{lst:r3b}
        \begin{minted}{bash}
        router BGP 100
        neighbor 192.168.100.1 remote-as 100
        neighbor 192.168.100.1 update-source lo 0
        \end{minted}
    \end{listing}

\newpage

\section{CM 2 - 8 septembre 2022}
\subsection{MPLS}
    \subsubsection{Définition}
    En language informatique et dans le domaine des télécommunications, 
    le MPLS (MultiProtocol Label Switching) est un mécanisme de transport 
    des données en réseau extrêmement modulaire, censé répondre aux difficultés 
    posées par les réseaux actuels qui transmettent des paquets IP (Internet 
    Protocol). Il s’agit en réalité d’un type d’architecture réseau étudié 
    par l’IETF. Il est censé étendre les fonctionnalités des infrastructures 
    IP actuelles et en résoudre les problèmes. En effet, la méthode de routage 
    employée aujourd’hui de type « unicast par saut » ( de l’émetteur vers le 
    récepteur), manque de flexibilité du fait de certaines restrictions 
    inhérentes à sa structure.\\

    Dans cette optique, le MPLS doit jouer un rôle important dans le routage, 
    la communication et le transfert de paquets au travers d’une infrastructure 
    réseau innovante. Elle devra prendre en compte à la fois les besoins de 
    service et les utilisateurs du réseau.\\

    Chaque entité n’aura ainsi besoin que d’une connexion vers un fournisseur 
    de service unique MPLS. Chaque noeud MPLS fonctionne aussi comme un routeur 
    IP, qui utilise des protocoles de routage IP afin d’échanger avec les 
    routeurs voisins.\\

    \subsubsection{Explication}
    L'intérêt de MPLS est de pouvoir faire du multiplexage de flux. Il vient comme
    base pour des services avancés. On en a besoin par exemple\\

    MPLS agit comme un en-tête sur une trame. MPLS est un protocole on peut 
    qualifier de 2.5. Le fournisseur d'accès possèdent des routeurs à la frontière
    de son système et à l'intérerieur de son réseau. 
\newpage

\subsection{LDP}
    \subsubsection{Définition}
    Le protocole LDP (Label Distribution Protocol) est un protocole permettant de 
    distribuer des étiquettes dans des applications non conçues pour le trafic. Le 
    LDP permet aux routeurs d’établir des chemins de commutation d’étiquettes (LSP) 
    via un réseau en mappant les informations de routage de la couche réseau 
    directement aux chemins commutés par couche de liaison de données.\\

    \subsubsection{Explication}
    Ces LSP peuvent avoir un point de terminaison à un voisin directement connecté 
    (comparable au transfert IP saut par saut) ou à un nœud de sortie du réseau, 
    ce qui permet de passer par tous les nœuds intermédiaires. Les LSP établies par 
    LDP peuvent également passer par des LSP conçus pour le trafic créés par RSVP.\\

    Le LDP associe une classe de transfert (FEC) à chaque LSP qu’il crée. Le FEC 
    associé à une LSP spécifie quels paquets sont mappés à ce LSP. Les LSP sont 
    étendus via un réseau car chaque routeur choisit le label annoncé par le 
    saut suivant pour le FEC et l’ajoute au label qu’il présente à tous les 
    autres routeurs. Ce processus forme un arbre de LSP qui converge vers le
    routeur de sortie\\


\newpage
\section{CM 3 - 8 septembre 2022 - M. Mura}
\subsection{Modulation analogique FM}
    \subsubsection{Récepteur homodyne}
    Le démodulateur veut récuperer la station 101,4 MHz, toutes les autres 
    stations sont donc considéré comme indésirable et d'après le schéma, une
    station occupe 120 KHz de BP.


    \subsubsection{Récepteur hétérodyne}
    Le récepteur hétérodyne est un récepteur qui utilise un oscillateur local. 
    Mais aussi c'est un récepteur conçu sur le principe du mélange de 
    fréquences, ou hétérodynage, pour convertir le signal reçu en une 
    fréquence intermédiaire plus basse qu'il est plus facile d'utiliser 
    que la fréquence reçue en direct.


    \subsubsection{Elimination de la fréquence image}


\newpage
\section{TD 2 - 13 septembre 2022}
\subsection{Exercice 1}
    \subsubsection{a)}
    Les avantages de MPLS sont que désormais l'inrerêt de MPLS n'est plus la 
    rapidité mais l'offre de services qu'il permet : MPLS-VPN, et le MPLS-TE, 
    qui ne sont plus réalisables avec le routage IP classique.\\

    \subsubsection{b)}
    Il existe différents types de LER : 
    \begin{itemize}
        \item LER-PE : (Label Edge Router-Provider Edge) Il se trouve à la 
        périphérie du réseau du FAI et il a un rôle important : il reçoit et 
        ajoute aux paquets ou trames du client une étiquette MPLS et les 
        transmet vers le cœur du réseau.\\
        \item LER-CE : (Label Edge Router-Customer Edge) C’est l’équipement 
        d’interconnexion entre le réseau du client et celui du FAI. Il peut 
        être de L2 ou L3 et il n’est pas concerné par MPLS.\\
        \item LER-P : (Label Edge Router-Provider) Il se trouve dans le cœur. 
    \end{itemize}

    \subsubsection{c)}
    La principale fonctionnalité assurée par les LSR est d'être connecté aux 
    PE et aux autres routeurs P. Il a pour rôle de commuter les paquets en 
    fonction de leurs étiquettes. Appelé aussi un routeur de transit.

\newpage

\section{TP 1/2/3 - 10/17/23 novembre 2022}
\subsection{Configuration des routeurs}
    \begin{verbatim}
        R1>enable
        R1#configure terminal
        R1(config)#interface fastEthernet 0/0
        R1(config-if)#ip address X.X.X.X Y.Y.Y.Y
        R1(config-if)#no shutdown
        R1(config-if)#exit
    \end{verbatim}

\subsection{Configuration du OSPF}
    \begin{verbatim}
        R6(config)#router ospf 1 (mettre le même numéro sur tous les routeurs)
        R6(config-router)#network 10.0.0.0 0.0.0.255 area 0
        R6(config-router)#router-id 6.6.6.6
        R6(config-router)#network 6.6.6.6 0.0.0.0 area 0
    \end{verbatim}


\subsection{Configuration du MPLS}
Pour activer le MPLS, il suffit de faire cette commande en choisissant les 
interfaces concercnés par le MMPLS :
    \begin{verbatim}
        R1(config)#mpls ip
        R1(config)#interface ethernet 0/0
        R1(config-if)#mpls ip
    \end{verbatim}

La commande permettant de vérifier les interfaces où MPLS est activé est :
    \begin{verbatim}
        R1#show mpls interfaces
    \end{verbatim}


\subsection{Questions du TP 2}
La commmande \texttt{show mpls ldp discovery} permet de voir les voisins LDP. 
Lorsque l'on désactive l'interface lo0 du routeur, nous constatons qu'en faisant
de nouveau la même commande qu'avant nous avons. 

\newpage

\subsection{Questions du TP 3}
La commande permettant de voir les voisins LDP d'un routeur est :
    \begin{verbatim}
        R1#show mpls ldp neighbor
    \end{verbatim}

La commande qui permet de voir les labels que chaque routeur génère pour 
les préfixes qu’il connait est : 
    \begin{verbatim}
        R1#show mpls ldp binding
    \end{verbatim}
Nous obtenons du coup toutes les adresses de loopback et leurs labels.
Nous avons aussi les adresses de proximité et leurs labels.



Lors que je fais \texttt{traceroute} \textit{IP de destination} on voit bien
qu'un label est utilisé car j'obtiens en retour : 
\begin{verbatim}
    R1#traceroute 3.3.3.3
    type escape sequence to abort.
    Tracing the route to 3.3.3.3
    VRF info: (vrf in name/id, vrf out name/id)
    1 10.10.10.2 [MPLS: Label 20 Exp 0] 1 msec 0 msec 0 msec
    2 10.10.10.1 1 msec 1 msec *
\end{verbatim}


\newpage
\subsection{Configuration du VRF}
\begin{verbatim}
    vrf definition Client1
        rd 100:1
        route-target export 100:1
        route-target import 100:1
        !
        address-family ipv4
        exit-address-family
    !
    vrf definition Client2
        rd 100:2
        route-target export 100:2
        route-target import 100:2
        !
        address-family ipv4
        exit-address-family
    !
\end{verbatim}
On utilise ici \texttt{vrf definition} pour définir le nom du VRF. 
Ensuite, pour \texttt{rd}, \texttt{100} correspond au numéro d'AS et \texttt{1}
correspond au numéro de client que l'on veut utiliser. On réitère la même 
chose pour le deuxième client.





\end{document}